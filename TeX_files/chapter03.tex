\Exo{La constante de Pythagore}{Normandie (Caen-Rouen)}{2017}{1}

\partie{A}

On note $L$ la longueur et $l$ la largeur.

\subExo{1}
D'après l'énoncé, un rectangle diagonal est tel que :
\begin{equation*}
	\dfrac{L}{l} = \dfrac{l}{\dfrac{L}{2}}
\end{equation*}

On peut donc dire que :
\begin{equation*}
	\begin{split}
		\dfrac{L}{l} & = \dfrac{l \cdot 2}{L} \\
		 \dfrac{L^2}{l} & = 2l \\
		 L^2 & = 2l^2 \\
		 \dfrac{L^2}{l^2} & = 2 \\
		 \dfrac{L}{l} & = \sqrt 2 \\
	\end{split}
\end{equation*}

\subExo{2}
On note $A(n)$ l'aire d'une feuille $A_n$, ainsi que $L_n$ et $l_n$ les longueurs et largeurs d'une feuille $A_n$. On a:
\begin{equation*}
	A(n) = L_n \cdot l_n
\end{equation*}

On sait que $A(0) = 1$. Trouvons $L_0$ et $l_0$ :
\begin{equation}
	\begin{split}
		L_0 \cdot l_0 & = 1 \\
		L_0 & = \dfrac 1 {l_0}
	\end{split}
\end{equation}
\begin{align*}
	{L_0}^2 &= \dfrac{L_0}{l_0} & \dfrac{L_0}{l_0}&=\dfrac 1 {{l_0}^2} \\
	{L_0}^2 &= \sqrt 2 & \sqrt{2}&=\dfrac 1 {{l_0}^2} \\
	L_0 &= \sqrt{\sqrt{2}} & l_0&=\dfrac{1}{\sqrt{\sqrt{2}}} \\
	& & l_0 &= \dfrac {\sqrt{\sqrt{2}}} {\sqrt{2}} \\
	& & l_0 &= \dfrac{\sqrt{2}\cdot\sqrt{ \sqrt{2}}}{\sqrt{2}\cdot\sqrt{2}}\\
	& & l_0 &= \dfrac{\sqrt{2\sqrt{2}}}{2}
\end{align*}

\partie{B}