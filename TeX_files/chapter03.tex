\Exo{La constante de Pythagore}{Normandie (Caen-Rouen)}{2017}{1}

\partie{A}

On note $L$ la longueur et $l$ la largeur.

\subExo{1}
D'après l'énoncé, un rectangle diagonal est tel que :
\begin{equation*}
	\dfrac{L}{l} = \dfrac{l}{\dfrac{L}{2}}
\end{equation*}

On peut donc dire que :
\begin{equation*}
	\begin{split}
		\dfrac{L}{l} & = \dfrac{l \cdot 2}{L} \\
		 \dfrac{L^2}{l} & = 2l \\
		 L^2 & = 2l^2 \\
		 \dfrac{L^2}{l^2} & = 2 \\
		 \dfrac{L}{l} & = \sqrt 2 \\
	\end{split}
\end{equation*}

\subExo{2}
On note $A(n)$ l'aire d'une feuille $A_n$, ainsi que $L_n$ et $l_n$ les longueurs et largeurs d'une feuille $A_n$. On a:
\begin{equation*}
	A(n) = L_n \cdot l_n
\end{equation*}

On sait que $A(0) = 1$. Trouvons $L_0$ et $l_0$ :
\begin{equation}
	\begin{split}
		L_0 \cdot l_0 & = 1 \\
		L_0 & = \dfrac 1 {l_0}
	\end{split}
\end{equation}
\begin{align*}
	{L_0}^2 &= \dfrac{L_0}{l_0} & \dfrac{L_0}{l_0}&=\dfrac 1 {{l_0}^2} \\
	{L_0}^2 &= \sqrt 2 & \sqrt{2}&=\dfrac 1 {{l_0}^2} \\
	L_0 &= \sqrt{\sqrt{2}} & l_0&=\dfrac{1}{\sqrt{\sqrt{2}}} \\
	& & l_0 &= \dfrac {\sqrt{\sqrt{2}}} {\sqrt{2}} \\
	& & l_0 &= \dfrac{\sqrt{2}\cdot\sqrt{ \sqrt{2}}}{\sqrt{2}\cdot\sqrt{2}}\\
	& & l_0 &= \dfrac{\sqrt{2\sqrt{2}}}{2}
\end{align*}

\partie{B}

\subExo{1}
Cas pair au carré :

Si la division en facteurs premiers d'un nombre contient $2$, alors son carré contient aussi $2$. Ce carré est donc pair.
\newline

\par Cas impair au carré :

Cette fois, la division en facteurs premiers ne contient pas de $2$. Le carré n'en contient pas non plus, le carré est donc impair.

\subExo{2}
Supposons qu'il existe $a$ et $b$ tels que $\frac a b = \sqrt{2}$

\subsubExo{a}
\begin{equation*}
	\begin{split}
		\dfrac{a^2}{b^2} &= 2 \\
		a^2 & = 2b^2
	\end{split}
\end{equation*}

On sait que $a^2$ est divisible par deux car si $a = 2n$, $a$ est pair. Or la racine d'un nombre pair est paire. Donc $\sqrt{a^2} = a$ est pair.

\subsubExo{b}
Si un carré est pair, sa décomposition en facteurs premiers contient au moins deux $2$. $a^2$ est pair, donc $\frac {a^2} 2$ est aussi pair. Or, $\frac{a^2}{2} = b^2$. $b^2$ est pair, donc $b$ est pair.

\subsubExo{c}
\begin{definition}
	Un nombre rationnel peut être écrit sous la forme d'une fraction \textbf{irréductible} $\frac{p}{q}$ avec $p$ et $q$ des entiers relatifs.
\end{definition}

On sait que $a$ et $b$ sont pairs, ce qui signifie qu'ils sont tous deux divisibles par 2. La définition dit que $\frac a b$ doit être irréductible, mais ici $\frac{a}{b}$ est réductible par 2. On en conclut que $\sqrt{2}$ n'est par un nombre rationnel.

\subExo{3}
Le nombre $\frac{22\,619\,537}{15\,994\,428}$ est rationnel, mais $\sqrt{2}$ n'est pas rationnel. On peut donc dire $\sqrt{2} \neq \frac{22\,619\,537}{15\,994\,428}$