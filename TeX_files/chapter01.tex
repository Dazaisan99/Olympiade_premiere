\Exo{Le fonctionnement d'un ordinateur}{Amiens}{2017}{4}


\subExo 1
\begin{center}
	\begin{tabular}{|c | c | c | c|} 
		\hline
		Action & $P_1$ & $P_2$ & Résultat \\ 
		\hline
		Situation initiale & $[a,b,c]$ & $[d,e]$ & Aucun \\ 
		\hline
		Lire $P_1$ & $[a,b,c]$ & $[d,e]$ & $c$ \\
		\hline
		Transfert de $P_1$ vers $P_2$ & $[a,b]$ & $[d,e,c]$ & Aucun \\
		\hline
		Dépiler $P_1$ & $[a]$ & $[d,e,c]$ & Aucun \\
		\hline
		Empiler la donnée "$f$" sur $P_1$ & $[a,f]$ & $[d,e,c]$ & Aucun \\
		\hline
		Transfert de $P_2$ vers $P_1$ & $[a,f,c]$ & $[d,e]$ & Aucun \\
		\hline
		Lire $P_2$ & $[a,f,c]$ & $[d,e]$ & $e$\\
		\hline
	\end{tabular}
\end{center}

\subExo 2
Soit $P_1 = [a,b,c]$

On ne peut pas lire $b$ sans recourir à une autre pile ni perdre des données : Il faut d'abord enlever $c$ soit avec "Transfert de $P_1$ vers $P_2$" soit avec "Dépiler $P_1$" avant de "Lire $P_1$", ce qui est à l'encontre des conditions.

\subExo 3
Dans ce cas, on peut commencer par faire "Transfert de $P_1$ vers $P_2$", ce qui donne\footnote{Si $P_2$ est vide} :

\[P_1 = [a,b]\]
\[P_2 = [c]\]

On peut alors "Lire $P_1$", ce qui donne $b$.

\subExo 4
Soient $P_1 = [a,b]$, $P_2 = []$, $P_3=[]$

\begin{center}
	\begin{tabular}{| c | c | c | c |}
		\hline
		Action & \pun & \pdeux & \ptrois \\
		\hline
		Situation initiale & $[a,b]$ & $[]$ & $[]$ \\
		\hline
		Transfert de \pun vers \ptrois & $[a]$ & $[]$ & $[b]$ \\
		\hline
		Transfert de \pun vers \pdeux & $[]$ & $[a]$ & $[b]$ \\
		\hline
		Transfert de \ptrois vers \pun & $[b]$ & $[a]$ & $[]$ \\
		\hline
		Transfert de \pdeux vers \pun & $[b,a]$ & $[]$ & $[]$ \\
		\hline
	\end{tabular}
\end{center}

\subExo 4
\begin{center}
	\begin{tabular}{| c | c | c | c |}
		\hline
		Action & \pun & \pdeux & \ptrois \\
		\hline
		Situation initiale & $[a,b,c]$ & $[]$ & $[]$ \\
		\hline
		Transfert de \pun vers \ptrois & $[a,b]$ & $[]$ & $[c]$ \\
		\hline
		Transfert de \pun vers \pdeux & $[a]$ & $[b]$ & $[c]$ \\
		\hline
		Transfert de \pun vers \pdeux & $[]$ & $[b, a]$ & $[c]$ \\
		\hline
		Transfert de \ptrois vers \pun & $[c]$ & $[b,a]$ & $[]$ \\
		\hline
		Transfert de \pdeux vers \ptrois & $[c]$ & $[b]$ & $[a]$ \\
		\hline
		Transfert de \pdeux vers \pun & $[c,b]$ & $[]$ & $[a]$ \\
		\hline
		Transfert de \ptrois vers \pun & $[c,b,a]$ & $[]$ & $[]$ \\
		\hline
	\end{tabular}
\end{center}