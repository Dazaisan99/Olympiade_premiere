\Exo{Palindromes binaires}{Lyon}{2017}{2}

\partie{1}

\subExo 1
\begin{equation*}
	\begin{split}
		135 & = 128 + 4 + 2 + 1 \\
		    & = 2^7 + 2^2 + 2^1 + 2^0 \\
		    & = (10000111)_2
	\end{split}
\end{equation*}
Le nombre $135$ se note donc $10000111$ dans le système binaire.

\subExo 2
\begin{equation*}
	\begin{split}
		(101011)_2 & = 2^5 + 2^3 + 2^1 + 2^0 \\
				   & = 32 + 8 + 2 + 1 \\
				   & = 43
	\end{split}
\end{equation*}
Donc $101011$ en binaire est $43$ en décimal.

\subExo 3\subsubExo{a}
Cas $N=6$:\par
Le reste de la division euclidienne de $N$ par $2$\footnote{noté par la suite $N\mod 2$} est $0$. $0$ est donc affiché.
Ensuite, on affecte à $N$ $6 \div 2 = 3$\footnote{Le quotient de la division euclidienne de $6$ par $2$}. On affiche $3 \mod 2 = 1$ et on affecte à $N$ $3\div 2 = 1$, puis on affiche $1 \mod 2 = 0$ et on affecte à $N$ $1 \div 2 = 0$. Comme $N = 0$. Le programme s'arrête.

Les nombres affichés sont donc $0, 1, 1$. \newline

Cas $N=53$ :\par

Les nombres affichés sont $1,0,1,0,1,1$

\subsubExo b
On sait que $6 = (110)_2$ et $53=(110101)_2$. On en conclut que le programme affiche les chiffre de $N$ converti en binaire à l'envers.

\subExo 4
On note $n$ le nombre de chiffres dans un nombre et $p$ la place d'un chiffre dans un nombre. On a : TODO : Preuve

\[2^n = (\sum_{p=1}^n 2^{n-p})+1\]

Or,
\begin{equation*}
	\begin{split}
		\underbrace{(111\dots1)_2}_{\text{$n$ fois}} & = 2^{n-1} + 2^{n-2}\dots2^{n-n} \\
				& = \sum_{p=1}^n 2^{n-p}
	\end{split}
\end{equation*}

On a donc bien $\underbrace{(111\dots1)_2}_{\text{$n$ fois}} = 2^n - 1$.

\subExo 5
On a autant de couleurs possibles sur $24$ bits que de nombre que l'on peut créer avec ces $24$ bits. Or, on a $2^{24} =16\,777\,216$ combinaisons différentes. On peut donc créer $16\,777\,216$ couleurs différentes avec ce système.

\vspace{3em}
\partie 2

\subExo {1}
Les années palindromes binaires entre $1$ et $129$ sont :
\begin{equation*}
	\begin{split}
		1 & = (1)_2 \\
		3 & = (11)_2 \\
		5 & = (101)_2 \\
		7 & = (111)_2 \\
		9 & = (1001)_2 \\
		15 & = (1111)_2 \\
		17 & = (10001)_2 \\
		21 & = (10101)_2 \\
		27 & = (11011)_2 \\
		31 & = (11111)_2 \\
		33 & = (100001)_2 \\
		45 & = (101101)_2 \\
		51 & = (110011)_2 \\
		63 & = (111111)_2 \\
		65 & = (1000001)_2 \\
		73 & = (1001001)_2 \\
		85 & = (1010101)_2 \\
		93 & = (1011101)_2 \\
		99 & = (1100011)_2 \\
		107 & = (1101011)_2 \\
		119 & = (1110111)_2 \\
		127 & = (1111111)_2 \\
		129 & = (10000001)_2
	\end{split}
\end{equation*}

\subExo 2
$2017 = (11111100001)_2$. Ce nombre n'est donc un palindrome binaire, car retourné il vaut $7231$

\subExo 3
La prochaine année palidrome binaire est $2047 = (11111111111)_2$

\subExo 4
On utilise les résultats du 1.

\begin{itemize}
	\item Palindromes à 3 chiffres : $2$ ($101$, $111$)
	\item Palindromes à 4 chiffres : $2$ ($1001$, $1111$)
	\item Palindromes à 5 chiffres : $4$ ($10001$, $10101$, $11011$, $11111$)
	\item Palindromes à 6 chiffres : $4$ ($100001$, $101101$, $110011$, $111111$)
	\item Palindromes à 7 chiffres : $8$ ($1000001$, $1001001$, $1010101$, $1011101$, $1100011$, $1101011$, $1110111$, $1111111$)
\end{itemize}

\subExo{5}
\subsubExo{a}

TODO: Finir l'exo